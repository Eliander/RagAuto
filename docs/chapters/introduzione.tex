%! Author = eliander
%! Date = 24/01/20
%! TEX encoding = UTF-8
%! TEX root = main.tex
\chapter{Introduzione}
\section{Consegna del progetto}
Implementare in un linguaggio di programmazione a scelta la procedura di decisione,
basata su chiusura di congruenza, della soddisfacibilità di un insieme di letterali nel
frammento senza quantificatori dell'unione delle teorie dell'uguaglianza, delle liste
non vuote possibilmente cicliche, e degli array senza estensionalità. Si implementi
l'algoritmo di chiusura di congruenza su grafo diretto aciclico, considerando le varianti
viste in classe (e.g., lista proibita, scelta non arbitraria del rappresentante della classe
generata da union, aggiornamento del campo find di tutti i nodi di una classe).
\section{Scelta del linguaggio per lo sviluppo}
Ottenuta la consegna, il primo dubbio da risolvere era relativo al linguaggio in cui implementare il solutore;
l'unico requisito per il linguaggio era che fosse eseguibile su Linux. Degli aspetti da non sottovalutare erano
la velocità di sviluppo (visto il tempo a disposizione) e la velocità di esecuzione (che sarebbe potuta essere un
parametro di valutazione). I linguaggi presi in considerazione erano tre:
\begin{itemize}
    \item Java
    \item Python
    \item Dart
\end{itemize}
JAVA, il linguaggio che conosco meglio, è abbastanza veloce in esecuzione ma avrei dovuto spendere troppo tempo per
strutturare il codice. Dart, simile a JAVA come prestazioni, avrebbe potuto rappresentare una scelta interessante: la
compatibilità sarebbe stata garantita tramite un'interfaccia web e questo mi avrebbe anche agevolato nel creare una UI;
putroppo la mia conoscenza di Dart era troppo bassa per cimentarmi in un progetto come questo. Python non è certo il
linguaggio più veloce, ma sicuramente è uno dei più versatili. La velocità di sviluppo e di debug sono alte ed esistono
parecchie librerie per gestire la parte grafica. Ho quindi scelto di sviluppare il progetto in Python, nello specifico
usando al versione 3.7.
