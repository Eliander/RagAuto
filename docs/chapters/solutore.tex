%! Author = eliander
%! Date = 24/01/20
%! TEX encoding = UTF-8
%! TEX root = main.tex
\chapter{Analisi del solutore}
\section{L'idea alla base}
Il progetto, fin dall'inizio, è stato pensato diviso in due parti: il solutore ed il parser.
Prima di analizzare il solutore, è necessario soffermarsi sulla struttura base dell'algoritmo
di congruence closure: il \textit{Node}. La classe \texttt{Node.py} definisce il mattone fondamentale
dell'algoritmo; i principali elementi della classe sono:
\begin{itemize}
    \item \texttt{int} \textbf{id}: id numerico del nodo
    \item \texttt{String} \textbf{fn}: simbolo della funzione del nodo
    \item \texttt{List<int>} \textbf{args}: id dei nodi argomenti
    \item \texttt{int} \textbf{find}: id del rappresentante della classe a cui appartiene il nodo
    \item \texttt{Set<int>} \textbf{ccpar}: id dei nodi genitori
\end{itemize}
I nodi vengono creati durante la fase di preprocessing dal parser: in questo modo, il solutore
deve preoccuparsi solamente di eseguire la chiusura di congruenza tra gli elementi.
\section{Il solutore}
Il solutore consiste in una classe, \texttt{Graph.py}, contenente i metodi necessari all'implementazione
della chiusura di congruenza; i metodi presenti sono:
\begin{center}
    \begin{tabular}{|m{6cm}|m{6cm}|}
        \hline
        \textbf{Metodo} & \textbf{Descrizione} \\ \hline
         \textbf{node}(\texttt{int id}): \texttt{Node n} & Restituisce un nodo dato l'id. \\ \hline
         \textbf{find}(\texttt{int id}): \texttt{int id} & Restituisce il find del nodo: se questo è diverso
                    da se stesso, chiama la funzione \textbf{find} sul nuovo id. \\ \hline
         \textbf{ccpar}(\texttt{int id}): \texttt{Set<int> id} & Restituisce gli id dei nodi genitori del rappresentante
                    della classe. \\ \hline
        \textbf{congruent}(\texttt{int id\_1, int id\_2}): \texttt{boolean result} & Confronta due nodi e restituisce
                    \texttt{true} se sono congruenti (simbolo di funzione ed argomenti uguali),
                    \texttt{false} altrimenti. \\ \hline
        \textbf{union}(\texttt{int id\_1, int id\_2}): \texttt{void} & Unisce due classi di congruenza,
                    accorpando i \textbf{ccpar} di un nodo nell'altro e cambiando il riferimento del \textbf{find} del
                    nodo svuotato verso l'altro nodo. \\ \hline
        \textbf{merge}(\texttt{int id\_1, int id\_2}): \texttt{void} & Metodo che permette di lanciare la \textbf{union}
                    sui termini che devono essere messi nella stessa classe di congruenza. \\ \hline
    \end{tabular}
\end{center}

