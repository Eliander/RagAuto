%! Author = eliander
%! Date = 24/01/20
%! TEX encoding = UTF-8
%! TEX root = main.tex
\chapter{Analisi del solutore}
\section{Il Node}
Prima di analizzare il solutore, è necessario soffermarsi sulla struttura base dell'algoritmo
di congruence closure: il \textit{Node}; definito nella classe \texttt{Node.py}, i principali elementi della classe sono:
\begin{itemize}
    \item \texttt{int} \textbf{id}: id numerico del nodo
    \item \texttt{String} \textbf{fn}: simbolo della funzione del nodo
    \item \texttt{List<int>} \textbf{args}: id dei nodi argomenti
    \item \texttt{int} \textbf{find}: id del rappresentante della classe a cui appartiene il nodo
    \item \texttt{Set<int>} \textbf{ccpar}: id dei nodi genitori
    \item \texttt{String} \textbf{sym}: simbolo del nodo
\end{itemize}
Di questi, l'attributo \textbf{sym} non era previsto dall'algoritmo; è un attributo di supporto utilizzato nella \textbf{congruent}
al posto di \textbf{fn}: ho preferito mantenere con \textbf{fn} la funzione del nodo (es. \textit{f(f(a))}) e con \textbf{sym}
il simbolo del nodo (\textit{f}).

\section{Il solutore}
Il solutore consiste in una classe, \texttt{Graph.py}, contenente i metodi necessari all'implementazione
della chiusura di congruenza. Questa classe è definita dai seguenti attributi:
\begin{itemize}
    \item \texttt{List<Node>} \textbf{DAG}: grafo formato dai nodi del problema
    \item \texttt{Map<int, String>} \textbf{index\_map}: mappa di supporto, utilizzata per ottenere il nome dei nodi dato
                                                        il loro indice
    \item \texttt{List<int>} \textbf{not\_in\_same\_set}: lista proibita, utilizzata per rendere più efficiente il programma
\end{itemize}
Oltre questi attributi, ne sono presenti altri tre utilizzati come supporto: \textbf{output}, una stringa utilizzata per
capire quando interrompere l'esecuzione (modificata dal controllo sulla \textit{lista proibita}) e \textbf{DETAILS} e
\textbf{PLOT}, parametri per stampare rispettivamente i dettagli dell'esecuzione e i grafici (inseriti a questo livello
per un problema di dipendenze circolari col Main).
I metodi presenti nella classe sono quelli per effettuare la chiusura di congruenza:
\begin{center}
    \label{tab:methods}
    \begin{tabular}{|m{6cm}|m{6cm}|}
        \hline
        \textbf{Metodo} & \textbf{Descrizione} \\ \hline
         \textbf{node}(\texttt{int id}): \texttt{Node n} & Restituisce un nodo dato l'id. \\ \hline
         \textbf{find}(\texttt{int id}): \texttt{int id} & Restituisce il find del nodo: se questo è diverso
                    da se stesso, chiama la funzione \textbf{find} sul nuovo id. \\ \hline
         \textbf{ccpar}(\texttt{int id}): \texttt{Set<int> id} & Restituisce gli id dei nodi genitori del rappresentante
                    della classe. \\ \hline
        \textbf{congruent}(\texttt{int id\_1, int id\_2}): \texttt{boolean result} & Confronta due nodi e restituisce
                    \texttt{true} se sono congruenti (simbolo di funzione ed argomenti uguali),
                    \texttt{false} altrimenti. \\ \hline
        \textbf{union}(\texttt{int id\_1, int id\_2}): \texttt{void} & Unisce due classi di congruenza,
                    accorpando i \textbf{ccpar} di un nodo nell'altro e cambiando il riferimento del \textbf{find} del
                    nodo svuotato verso l'altro nodo. \\ \hline
        \textbf{merge}(\texttt{int id\_1, int id\_2}): \texttt{void} & Metodo che permette di lanciare la \textbf{union}
                    sui termini che devono essere messi nella stessa classe di congruenza. \\ \hline
    \end{tabular}
\end{center}

\section{Euristiche applicate}
Tra le euristiche viste a lezione, ho scelto di implementare la \textit{forbidden list}: questa particolare euristica permette
di far terminare il programma prima dell'esecuzione di tutte le clausole nel caso in cui gli elementi della lista proibita
siano nella stessa classe di congruenza (abbiano quindi lo stesso campo \textbf{find}). Il controllo viene effettuato dopo ogni
iterazione dell'algoritmo. Un'altra euristica implementata (e poi rimossa) consisteva nell'\textit{aggiornamento ricorsivo del campo
find} di tutti gli elementi appartenenti all'insieme durante l'unione; questa variante è stata scartata in
quanto invece di ridurre il tempo di esecuzione, lo aumentava. La terza euristica applicata consiste nella \textit{scelta
non casuale del nodo rappresentante}: quando viene eseguita una \textbf{union} tra due nodi, il programma si preoccupa di contare
tutti i nodi del grafo che hanno come rappresentante di classe uno dei due \textbf{id} coinvolti. Una volta trovato, quello con
più ocorrenze diventa il rappresentante dell'altro. Questa euristica, nonostante non abbia praticamente modificato
le prestazioni (il rallentamento conseguitone è veramente minimo), è stata mantenuta perchè aiuta ad orientarsi meglio nel
flusso delle stampe: vedere spesso lo stesso \textbf{id} come referente della classe di congruenza mi è stato molto
d'aiuto, soprattutto in fase di debug.

