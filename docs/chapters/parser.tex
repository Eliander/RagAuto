%! Author = eliander
%! Date = 24/01/20
%! TEX encoding = UTF-8
%! TEX root = main.tex
\chapter{Il parser}
\section{Breve analisi}
Il progetto, fin dall'inizio, è stato strutturato in due parti: il solutore ed il parser. Il solutore ha il compito di
eseguire l'algoritmo di chiusura di congruenza mentre il parser si occupa della fase di preprocessing dell'input e della creazione
dei nodi. Nonostante il parser non sia la parte principale del progetto, trovo doveroso spendere alcune parole al riguardo.
\section{Preprocessing per le teorie di liste e array}
La classe \texttt{InputParser.py} si occupa del preprocessing dell'input: per poter applicare l'algoritmo di congruence
closure anche alle liste ed agli array, infatti, è necessario aggiungere o rimuovere equazioni a seconda dei termini presenti.
Se la formula da analizzare contiene un termine \textit{!atom(x)}, è necessario sostituirlo con un termine
\textit{cons($x_u$, $x_v$) = x}. Per implementare il \textbf{merge} relativo al \textit{cons}, il parser aggiunge due equazioni
\textit{$x_u$ = car(cons($x_u$, $x_v$))} e \textit{$x_v$ = cdr(cons($x_u$, $x_v$))}. Altro caso si verifica quando l'equazione
analizzata contiene \textit{store} o \textit{select}; se per la select basta una normale sostituzione (es. da \textit{select(a, j)}
a \textit{$f_a$(j)}), per la store le cose sono più complicate: il parser effettua una analisi dei casi, andando a generare
tutte le possibili varianti di indici e valori. Questo particolare punto mi ha costretto a cambiare in corso d'opera
la struttura stessa del parser: l'input che arrivava era sempre una stringa ma gli insiemi di clausole a cui applicare la
chiusura di congruenza diventavano più d'uno.
\section{Cenni sul parsing}
L'idea alla base consiste nell'analizzare le stringe mediante il conteggio delle \textit{parentesi}: valutando le aperture
e le chiusure, l'algoritmo estrae i termini più esterni; fatto ciò, se sono presenti altre parentesi, ricorsivamente esplora
i \textit{sottotermini}. Infine ottiene i \textit{termini singoli} (tra parentesi e non). I nodi creati in questo modo
vengono passati al solutore che deve preoccuparsi solamente di eseguire la chiusura di congruenza tra gli elementi.